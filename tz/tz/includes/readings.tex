\section{Источники, использованные при разработке}

\begin{enumerate}
    \item ГОСТ 19.101-77 Виды программ и программных документов. // Единая система программной документации. – М.: ИПК Издательство стандартов, 2001.
    \item ГОСТ 19.102-77 Стадии разработки. // Единая система программной документации. – М.: ИПК Издательство стандартов, 2001.
    \item ГОСТ 19.103-77 Обозначения программ и программных документов. // Единая система программной документации. – М.: ИПК Издательство стандартов, 2001
    \item ГОСТ 19.104-78 Основные надписи. // Единая система программной документации. – М.: ИПК Издательство стандартов, 2001.
    \item ГОСТ 19.105-78 Общие требования к программным документам. // Единая система программной документации. – М.: ИПК Издательство стандартов, 2001.
    \item ГОСТ 19.106-78 Требования к программным документам, выполненным печатным способом. // Единая система программной документации. – М.: ИПК Издательство стандартов, 2001
    \item ГОСТ 19.404-79 Пояснительная записка. Требования к содержанию и оформлению. // Единая система программной документации. – М.: ИПК Издательство стандартов, 2001.
    \item Загрузка файлов [Электронный ресурс] / Spring-Projects. Режим доступа: \href{https://spring-projects.ru/guides/uploading-files/}{https://spring- projects.ru/guides/uploading-files/}, свободный. (Дата обращения: 15.05.2020)
    \item Как установить и настроить PostgreSQL в MacOS [Электронный ресурс] / 900913. Режим доступа: \url{https://900913.ru/note/b/postgresql-macos-9da176/}, свободный. (Дата обращения: 15.05.2020)
    \item Курс по Spring / ВТБ // [Электронный ресурс]: Google Drive. Режим доступа: \url{https://drive.google.com/drive/folders/1PQspMs1gm8aIFNua9l-wDsjgPxidqfhC}, свободный. (Дата обращения: 15.05.2021)
    \item Тянем ролик с Youtube и раздаем по WebRTC в реалтайме [Электронный ресурс] / Flashphoner. Режим доступа: \href{https://flashphoner.com/tyanem-rolik-s-youtube-i-razdaem-po-webrtc-v-realtajme/?lang=ru}{https://flashphoner.com/tyanem-rolik-s-youtube-i-razdaem-po-webrtc-v-realtaj me/?lang=ru}, свободный. (Дата обращения: 15.05.2020)
    \item Adaptive HTTP Streaming Technologies: HLS vs. DASH / Tech Blog // [Электронный ресурс]: StriveCast. Режим доступа: \url{https://strivecast.com/hls-vs-mpeg-dash/}, свободный. (Дата обращения: 15.05.2021)
    \item Create Multiple tables [Электронный ресурс] / LaunchSchool. Режим доступа: \url{https://launchschool.com/books/sql_first_edition/read/multi_tables}, свободный. (Дата обращения: 15.05.2020)
    \item Docker документация [Электронный ресурс] / Docker. Режим доступа: \url{https://docs.docker.com/}, свободный. (Дата обращения: 15.05.2020)
    \item ffmpeg документация [Электронный ресурс] / ffmpeg. Режим доступа: \url{https://ffmpeg.org/ffmpeg.html}, свободный. (Дата обращения: 15.05.2020)
    \item GitHub [Электронный ресурс]. Режим доступа: \url{https://github.com/}, свободный. (Дата обращения: 15.05.2020)
    \item How do I use ffmpeg to get the video resolution? / Vladimir Stazhilov // [Электронный ресурс]: SuperUser. Режим доступа: \href{https://superuser.com/questions/841235/how-do-i-use-ffmpeg-to-get-the-video-resolution}{https://superuser.com/questions/841235/how-do-i-use- ffmpeg-to-get-the-video-resolution}, свободный. (Дата обращения: 15.05.2021)
    \item How to create .mpd or .m3u8 video file on the server using FFMPEG for Adaptive Streaming / Mayur Solanki // [Электронный ресурс]: Medium. Режим доступа: \href{https://mayur-solanki.medium.com/how-to-create-mpd-or-m3u8-video-file-from-server-using-ffmpeg-97e9e1fbf6a3}{https://mayur-solanki. medium.com/how-to-create-mpd-or-m3u8-video-file-from-server-using-ffmpeg-97e9e1fbf6a3}, свободный. (Дата обращения: 15.05.2021)
    \item How to read ffmpeg response from java and use it to create a progress bar? / shalki // [Электронный ресурс]: StackOverflow. Режим доступа: \href{https://stackoverflow.com/questions/10927718/how-to-read-ffmpeg-response-from-java-and-use-it-to-create-a-progress-bar}{https://stackoverflow.com/questions/10927 718/how-to-read-ffmpeg-response-from-java-and-use-it-to-create-a-progress-bar}, свободный. (Дата обращения: 15.05.2021)
    \item How video streaming works on the web: An introduction / Paul Berberian // [Электронный ресурс]: Medium. Режим доступа: \href{https://medium.com/canal-tech/how-video-streaming-works-on-the-web-an-introduction-7919739f7e1}{https://medium.com/canal-tech/how-video-streaming-works-on-the-web-an-introduction-7919739f7e1}, свободный. (Дата обращения: 15.05.2021)
    \item Kotlin документация [Электронный ресурс] / JetBrains. Режим доступа: \url{https://kotlinlang.org/docs/}, свободный. (Дата обращения: 15.05.2020)
    \item PostgreSQL документация [Электронный ресурс] / PostgreSQL. Режим доступа: \url{https://www.postgresql.org/docs/}, свободный. (Дата обращения: 15.05.2020)
    \item Spring документация [Электронный ресурс] / Spring. Режим доступа: \url{https://spring.io/}, свободный. (Дата обращения: 15.05.2020)
    \item Spring Cloud Demo Project / gonwan // [Электронный ресурс]: GitHub. Режим доступа: \url{https://github.com/gonwan/spring-cloud-demo}, свободный. (Дата обращения: 15.05.2021)
    \item Spring Cloud Netflix Microservices -- start project (серия статей) -- часть 4 / Kirill Sereda // [Электронный ресурс]: Medium. Режим доступа: \url{https://medium.com/@kirill.sereda/spring-cloud-netflix-microservices-start-project-%D1%81%D0%B5%D1%80%D0%B8%D1%8F-%D1%81%D1%82%D0%B0%D1%82%D0%B5%D0%B9-%D1%87%D0%B0%D1%81%D1%82%D1%8C-4-d2137d19d783}, свободный. (Дата обращения: 15.05.2021)
    \item StackOverflow [Электронный ресурс]. Режим доступа: \url{https://stackoverflow.com/}, свободный. (Дата обращения: 15.05.2020)
    \item streaming an mkv file while processing with ffmpeg / Phani Rithvij // [Электронный ресурс]: StackOverflow. Режим доступа: \url{https://stackoverflow.com/questions/55460359/streaming-an-mkv-file-while-processing-with-ffmpeg}, свободный. (Дата обращения: 15.05.2021)
\end{enumerate}