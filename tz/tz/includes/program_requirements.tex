\section{Требования к программе}

\subsection{Требования к функциональным характеристикам}

Программа состоит из двух основных компонент: клиентской и серверной частей, между которыми должно быть налажено
взаимодействие.

\subsubsection{Требования к клиентской части}
Клиент должен иметь интерфейс, позволяющий пользователю взаимодействовать с программой с минимальной предварительной
подготовкой.
Дизайн интерфейса должен соответствовать современным тенденциям, обладать адаптивностью под различные характеристики
экранов.
Интерфейс должен менять свой стиль в зависимости от времени суток для имитации освещения в комнате.

Клиент реализует три основных экрана.
На каждом экране расположен свой набор элементов:\\

На главной странице:
\begin{enumerate}[noitemsep]
    \item Поле для загрузки файла.
\end{enumerate}

На странице комнаты:
\begin{enumerate}[noitemsep]
    \item Ссылка для подключения к комнате.
    \item Список участников комнаты.
    \item Текстовый чат пользователей с возможностью его чтения и отправки сообщений.
    \item Панель с всплывающими <<реакциями>>.
    \item Проигрыватель с видео.
\end{enumerate}

В проигрывателе с видео:
\begin{enumerate}[noitemsep]
    \item Кнопка изменения качества видео.
    \item Возможность отправлять сообщения в чат.
    \begin{enumerate}
        \item Обычный текстовый ввод сообщений;
        \item Отправка «реакций» в виде стикеров для быстрых эмоций.
    \end{enumerate}
    \item Отображение сообщений чата.
\end{enumerate}

\newpage

\subsubsection{Требования к серверной части}
На серверной части должен быть реализован алгоритм по преобразованию единого видеоролика в набор видеороликов меньшей длительности
(сегментов).
Также требуется добавить конвертацию видеоролика в видеоролики с меньшим качеством (Рис.~\ref{ris:server_converting}).

Алгоритм конвертации видеоролика следующий:
\begin{enumerate}
    \item Клиент загружает видеофайл на сервер.
    \item Сервер конвертирует видеофайл в видеофайлы меньших разрешений.
    \[ video1080 \Rightarrow videos = \{ video1080, video720, video480, video360, video240, video144 \} \]
    Данное действие требуется сделать для обеспечения возможности менять качество воспроизводимых роликов в плеере клиента,
    а также для сжатия самих файлов с целью ускорения кэширования видеороликов клиентом.
    \item Сервер разделяет каждый видеоролик из множества видеороликов \(videos\) на небольшие фрагменты (1 сек.).
    \[ video = part1 + part2 + \cdots, \;\;\; video \in videos \]
    Данное действие требуется сделать для обеспечения возможности измененять качество и настройки видео в процессе его воспроизведения
    (без повторного кэширования и приостановки воспроизведения).
\end{enumerate}

Требуется реализовать API для обеспечения взаимодействия сервера с клиентами.
Реализованные методы API и протоколы взаимодействия должны быть задокументированы.

Должно быть реализовано взаимодействие с базой данных для хранения данных о комнатах.

\newpage

\subsubsection{Требования к взаимодействию клиентской и серверной частей}
Взаимодействие между клиентом и сервером должно осуществляться посредством HTTP-запросов и WebSocket-подключений.

При получении HTTP-запроса (GET, POST, UPDATE, DELETE и т.д.) от клиента, сервер должен ответить сообщением в формате
JSON, содержащим необходимую информацию для работы клиента.

Для синхронизации видеопотока между разными клиентами используется протокол WebSocket.
В целях обеспечения наименьшей рассинхронизации видеопотока между клиентами, требуется хранить переменные \(diff\_sc\) и \(diff\_cs\) для
каждого клиента.
Данные переменные будут содержать в себе информацию о количестве затрачиваемого времени при передаче данных от сервера к клиенту или от клиента к серверу.

Формула для определения переменных \(diff\_sc\) и \(diff\_cs\): \begin{gather*}
                                                                    diff\_sc = time\_sc_2 - time\_sc_1\\
                                                                    diff\_cs = time\_cs_2 - time\_cs_1
\end{gather*}, где:
\begin{itemize}[noitemsep]
    \item[--] \(time\_sc_1\) — время отправки сообщения сервером;
    \item[--] \(time\_sc_2\) — время получения сообщения клиентом;
    \item[--] \(diff\_sc\) — разница между временем отправки и временем получения при передаче сообщения от сервера к клиенту;
    \item[--] \(time\_cs_1\) — время отправки сообщения клиентом;
    \item[--] \(time\_cs_2\) — время получения сообщения сервером;
    \item[--] \(diff\_cs\) — разница между временем отправки и временем получения при передаче сообщения от клиента к серверу.
\end{itemize}

Так как возможно подключение нескольких клиентов, требуется хранить содержимое значений \(diff\_sc_i\) и \(diff\_cs_i\) для всех \(n\) клиентов.

Алгоритм (Рис.~\ref{ris:interaction_format}) синхронизации видео единый:
\begin{enumerate}
    \item Клиент \(k\) отправляет серверу запрос на действие \(d\) (перемотку/приостановку/возобновление) видео.
    \item Сервер отправляет всем клиентам сообщение с текущим серверным временем.
    Клиенты принимают значение и считают разницу \(diff\_sc_i\) с учётом своего времени.
    Затем клиенты отправляют посчитанную разницу времени в миллисекундах обратно серверу, а также отправляют текущее клиентское время.
    С учётом полученной информации сервер считает разницу \(diff\_cs_i\).
    \item Сервер отправляет всем клиентам команду выполнить действие \(d\) и передаёт каждому клиенту задержку \(delay_i\),
    которая считается по формуле: \[ delay_i = \max(diff\_sc_1, \ldots, diff\_sc_n) - diff\_sc_i + diff\_cs_k, \;\;\; i \in [1, \ldots, n] \]
\end{enumerate}

Значение \(delay_i\) требуется по-разному использовать в различных ситуациях.
При организации совместного просмотра фильма возможны следующие сценарии:
\begin{itemize}
    \item[--] \textbf{Перемотка} видео на позицию \(t\) мс одним из клиентов.

    При получении такой команды все клиенты (кроме клиента-инициатора) выполняют перемотку видео на позицию \((t + delay_i)\) мс.
    \item[--] \textbf{Приостановка} видео одним из клиентов.

    При получении такой команды все клиенты (кроме клиента-инициатора) приостанавливают воспроизведение и выполняют перемотку видео на \(delay_i\) мс назад.
    \item[--] \textbf{Возобновление} видео одним из клиентов.

    При получении такой команды все клиенты (кроме клиента-инициатора) возобновляют воспроизведение и выполняют перемотку видео на \(delay_i\) мс вперёд.
\end{itemize}

\begin{figure}[p]
    \centering
    \includegraphics[width=0.97\linewidth]{../images/server_converting.png}
    \caption{Алгоритм конвертации видео}
    \label{ris:server_converting}
\end{figure}

\begin{figure}[p]
    \centering
    \includegraphics[width=0.97\linewidth]{../images/interaction_format.png}
    \caption{Алгоритм синхронизации видеопотока}
    \label{ris:interaction_format}
\end{figure}

\newpage

\subsubsection{Требования к организации входных данных}
Входные данные программы должны быть организованы в виде вводимого в специальную форму текста или файла,
соответствеющего определённому шаблону.
Данные, вводимые вручную, проверяются на корректность после попытки сохранения;
данные, вводимые из файла, проверяются в ходе анализа и размещения данных.

\subsubsection{Требования к организации выходных данных}
Выходные данные программы должны быть организованы в виде визуальных эффектов интерфейса, текстов, картинок и
видеороликов, расположенных на WEB-странице.

Видеоролики должны делиться на фрагменты, которые в последствии WEB-проигрыватель должен склеивать в полноценное видео.

\subsection{Требования к надёжности}

\subsubsection{Требования к обеспечению надёжного (устойчивого) функционирования программы}

Разрабатываемый WEB-клиент должен:
\begin{enumerate}
    \item не завершаться аварийно при возникающих ошибках.
\end{enumerate}

Разрабатываемый сервер должен:
\begin{enumerate}
    \item запрещать доступ к REST API методам для неавторизованных пользователей;
    \item не показывать информацию из посторонних общежитий для авторизованных пользователей;
    \item не показывать чужие обращения пользователю;
    \item быть устойчивым к атакам следующего типа:
    \begin{enumerate}
        \item Cross-Site Scripting (XSS),
        \item SQL Injection,
        \item Local File Inclusion (LFI),
        \item Distributed Denial of Service (DDoS).
    \end{enumerate}
\end{enumerate}

\subsubsection{Время восстановления после отказа}
Время восстановления после отказа, вызванного сбоем электропитания технических средств (иными внешними факторами),
не фатальным сбоем (не крахом) операционной системы, не должно превышать времени, необходимого на перезагрузку
операционной системы и запуск программы, при условии соблюдения условий эксплуатации технических и программных средств.

Время восстановления после отказа, вызванного неисправностью технических средств, фатальным сбоем (крахом) операционной
системы, не должно превышать времени, требуемого на переустановку программных средств.

\subsubsection{Отказы из-за некорректных действий оператора}
Во избежание возникновения отказов программы по причине некорректных действий оператора следует обеспечить работу
конечного пользователя без предоставления ему административных привилегий.

\subsection{Условия эксплуатации}

\subsubsection{Климатические условия}

Климатические условия сопадают с климатическими условиями эксплуатации устройства.

\subsubsection{Требования к пользователю}

Пользователь должен обладать базовыми навыками работы с одной из операционных систем: Windows, macOS, Linux, Android,
iOS, а также с одним из браузеров: Google Chrome, Firefox, Safari, Microsoft Edge.

\subsection{Требования к составу и параметрам технических средств}

\begin{itemize}
    \item[--] В состав технических средств WEB-клиента может входить портативный компьютер или мобильный телефон.
    Для корректной загрузки видеофайлов требуется стабильное интернет-подключение.
    \item[--] В состав технических средств сервера должен входить компьютер или система компьютеров.
    Допускается использование облачных сервисов.
\end{itemize}

\subsection{Требования к информационной и программной совместимости}

\subsubsection{Требования к программной совместимости}
Для корректной работы WEB-клиента должен использоваться один из следующих браузеров:
\begin{enumerate}[noitemsep]
    \item Google Chrome (86.0.4240.183+);
    \item Firefox (20.1+);
    \item Safari (14.0+);
    \item Microsoft Edge (86.0.622.63+).
\end{enumerate}

\subsubsection{Требования к исходным кодам и языкам программирования}

Исходные коды для WEB-клиента должны быть реализованы с использованием следующих языков и технологий:
\begin{enumerate}[noitemsep]
    \item HTML5 и CSS — для реализации графического представления клиента;
    \item JavaScript — для добавления динамики графическому интерфейсу;
    \item ReactJS или vue.js — для реализации архитектуры и программной логики WEB-клиента.
\end{enumerate}

Исходные коды для сервера должны быть реализованы с использованием следующих языков и технологий:
\begin{enumerate}[noitemsep]
    \item Java;
    \item Spring Boot — для реализации базовой архитектуры сервера;
    \item Spring Web — для реализации REST API методов и панели администрирования;
    \item Spring Thymeleaf — для миграций и версионирования базы данных;
    \item Hibernate — для работы с базой данных, для связки таблиц базы данных с классами Java;
    \item ffmpeg — для обработки видеофайлов: нарезка и изменение качества;
    \item PostgreSQL — использовать в качестве СУБД.
\end{enumerate}

\paragraph{Требования к взаимодействию клиентов с сервером}
Взаимодействие между WEB-клиентом и сервером должно происходить посредством HTTP-запросов и WebSocket-подключений.
При получении HTTP-запроса (GET, POST, UPDATE, DELETE и т.д.) от клиента, сервер должен ответить сообщением в формате
JSON, содержащим необходимую информацию для работы клиента.

\subsection{Требования к составу сетевых средств}

У устройства должен быть доступ к сети интернет для корректной работы WEB-клиента.

\subsection{Требования к маркировке и упаковке}

Сервер должен быть удалённо развёрнут на одном из облачных сервисов.
Для обеспечения переносимости требуется настроить контейнер Docker.

WEB-клиент публикуется на удалённом сервере с настроенным доменом «comnata.tv».