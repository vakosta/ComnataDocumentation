\documentclass{../includes/TechDocMultiAuthors}
\usepackage[T1]{fontenc}
\usepackage[utf8]{inputenc}
\usepackage{hyperref}
\usepackage{amssymb}
\usepackage{amsmath,amsthm,mathtools}

\setcounter{tocdepth}{2}
\setcounter{secnumdepth}{3}

\renewcommand{\cftsecleader}{\cftdotfill{\cftdotsep}}

\newcommand{\intro}[2]{
\stepcounter{section}
\section*{\hfillПРИЛОЖЕНИЕ #2}
\begin{center}
    \Large\bf{#1}
\end{center}
\markboth{\MakeUppercase{#1}}{}
\addcontentsline{toc}{section}{Приложение #2. #1}
}

\title{Приложение для совместного просмотра фильмов}
\author{Студент группы БПИ194}{В. А. Анненков}
\academicTeacher{Старший преподаватель департамента программной инженерии факультета компьютерных наук}{А. В. Поповкин}

\documentTitle{Программа и методика испытаний}
\documentCode{RU.17701729.02.06-01 51 01-1}

\begin{document}
    \maketitle

    \begin{abstract}
        Настоящая Программа и методика испытаний предназначена для правильной организации работы <<Приложения для совместного просмотра фильмов>>.

        Данная Программа и методика испытаний содержит следующие разделы: «Объект испытаний», «Цель испытаний», «Требования к программе», «Требования к программной документации», «Средства и порядок испытаний», «Методы испытаний».

        В разделе «Объект испытаний» указывают наименование, область применения и обозначение испытуемой программы.

        В разделе «Цель испытаний» должна быть указана цель проведения испытаний.

        В разделе «Требования к программе» должны быть указаны требования, подлежащие проверке во время испытаний и заданные в техническом задании на программу.

        В разделе «Требования к программной документации» должны быть указаны состав программной документации, предъявляемой на испытания, а также специальные требования, если они заданы в техническом задании на программу.

        В разделе «Средства и порядок испытаний» должны быть указаны технические и программные средства, используемые во время испытаний, а также порядок проведения испытаний.

        В разделе «Методы испытаний» должны быть приведены описания используемых методов испытаний.
    \end{abstract}
    \newpage

    \tableofcontents

    \section{Объект испытаний}

    \subsection{Наименование испытуемой программы}

    \subsubsection{Наименование программы на русском языке}

    Приложение для совместного просмотра фильмов.

    \subsubsection{Наименование программы на английском языке}

    Application for Collective Movie Watching.

    \subsection{Область применения испытуемой программы}

    <<Приложение для совместного просмотра фильмов>> может быть использовано пользователями для совместного просмотра фильмов или других видео на расстоянии.

    \section{Цель испытаний}

    Цель проведения испытаний -- проверка корректности работы программы, а также проверка соответствия разработканной программы функциональным требованиям и требованиям к надёжности, изложенным в документе <<Техническое задание>>.

    \section{Требования к программе}

    Программа обеспечивает возможность выполнения следующих функций:
    \begin{enumerate}
        \item Загрузка видеофайла для его дальнейшей обработки в формат HLS.
        \item Корректное получение обработанного видео в формате HLS.
        \item Получение актуальных данных о перемотке видео с других клиентов.
        Поддержание видео в актуальном состоянии.
        \item Получение актуальных данных о сообщениях в чате.
        \item Получение актуальных данных о <<реакциях>>.
    \end{enumerate}

    \section{Требования к программной документации}

    \subsection{Состав программной документации}

    В состав программной документации должны входить следующие компоненты:
    \begin{enumerate}
        \item Техническое задание (ГОСТ 19.201-78)
        \item Программа и методика испытаний (ГОСТ 19.301-78)
        \item Пояснительная записка (ГОСТ 19.404-79)
        \item Руководство оператора (ГОСТ 19.505-79)
        \item Текст программы (ГОСТ 19.401-78)
    \end{enumerate}

    \subsection{Специальные требования к программной документации}

    Документы к программе должны быть выполнены в соответствии с ГОСТ 19.106-78 и ГОСТами к каждому виду документа (см. п. 5.1.);

    Пояснительная записка должна быть загружена в систему Антиплагиат через LMS «НИУ ВШЭ».

    Документация и программа сдаются в электронном виде в формате .pdf или .docx. в архиве формата .zip или .rar;

    За один день до защиты комиссии все материалы курсового проекта:
    \begin{itemize}
        \item[--] техническая документация,
        \item[--] программный проект,
        \item[--] исполняемый файл,
        \item[--] отзыв руководителя,
        \item[--] лист Антиплагиата
    \end{itemize}
    должны быть загружены одним или несколькими архивами в проект дисциплины «Курсовой проект 2020-2021» в личном кабинете в информационной образовательной среде LMS (Learning Management System) НИУ ВШЭ.

    \section{Средства и порядок испытаний}

    \subsection{Технические средства, используемые во время испытаний}

    Во время проведения испытаний были использованы следующие технические средства:

    \begin{enumerate}
        \item 8 ГБ оперативной памяти;
        \item SSD диск объёмом 256 ГБ;
        \item процессор 2,3 GHz Quad-Core Intel Core i5;
        \item видеокарта Intel Iris Plus Graphics 655 1536 MB;
        \item монитор с расширением 2560 на 1600 точек;
        \item мышь;
        \item клавиатура.
    \end{enumerate}

    \subsection{Программные средства, используемые во время испытаний}

    Во время проведения испытаний были использованы следующие программные средства:
    
    \begin{enumerate}
        \item ОС Windows 10 и ОС macOS 11.2.3;
        \item браузеры Google Chrome 89.0.4389.114 и Safari 16610.4.3.1.7.
    \end{enumerate}

    \subsection{Порядок проведения испытаний}

    Порядок проведения испытаний должен включать:

    \begin{enumerate}
        \item проверку требований к функциональным характеристикам;
        \item проверку требований к формату входных и выходных данных;
        \item проверку требований к надёжности.
    \end{enumerate}

    \section{Методы испытаний}

    \subsection{Испытание клиента}

    \subsection{Испытание сервера}

    \registrationList
\end{document}
